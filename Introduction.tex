

\begin{Introduction}
In spite of advanced technologies and treatment, triple negative breast cancer still facing the problems of tumor recurrence and drug resistance.
For any given difference between the types of drug resistance, for example, the expression of a particular gene, it is assumed that differences arise deterministically or probabilistically in the configuration of transcription factors regulating the genes in the tumors. Cancer cells in distinct cell- states often exhibit important differences in functional properties depending on the which genes are turned on and off resulting in sensitive or resistant phenotype.

The most challenging analysis is to differentiate whether the change in gene expression leading to change in cellular state is stochastic\cite{raj2008nature} and random or its deterministic to produce the same output under similar environment.
Cancer cells in distinct cell- states often exhibit important differences in functional properties depending on which genes are turned on and off resulting in sensitive  or resistant phenotype.
Previously it is shown that unique cells within a population can exhibit fluctuations in expression of a group of genes, that could predict distinct phenotypes \cite{shaffer2019memory}.
Un-like genomic clones that could get selected in resistant phenotype \cite{salehi2020single}, we still are not clear whether the cell-states are acquainted for this kind of behaviour or the selected states are pre-existing in the cancer population and under continues pressure shows obvious dynamics or there is a transition from one state to another that ultimately gets selected over time. Because of difficulty to analyse longitudinal patient's samples for single cell gene expression and lack of multiple longitudinal pre-clinical breast cancer models, these questions remains unexplored.

The introduction of RNA velocity in single cells has opened up new ways of studying cellular differentiation. Time derivative of the gene expression state can be directly estimated by distinguishing unspliced and spliced mRNAs in common single-cell RNA sequencing protocols. RNA velocity is a high-dimensional vector that predicts the future state of individual cells on a timescale of hours.......
Here we set to examine three breast cancer patient derived xenografts (PDX) that were serially challenged for around 4-5 cycles with platinum compound until they started showing resistant behaviour. We wanted to understand the magnitude of fluctuations in gene expression from sensitive to resistant phenotype.

 
\end{Introduction}